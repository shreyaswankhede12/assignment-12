% Inbuilt themes in beamer
\documentclass{beamer}

% Theme choice:
\usetheme{CambridgeUS}

% Title page details: 
\title{Assignment 12\\Probability and Random Variables} 
\author{Shreyas Wankhede}
\date{\today}
\institute{IIT Hyderabad}
\logo{\large \LaTeX{}}


\begin{document}

% Title page frame
\begin{frame}
    \titlepage 
\end{frame}

% Remove logo from the next slides
\logo{}


% Outline frame
\begin{frame}{Outline}
    \tableofcontents
\end{frame}


% Lists frame
\section{Question}
\begin{frame}{Question}
\begin{block}{Question 9.47}
Show that if $R_x(\tau)=Ae^{jw_0\tau}$,then $R_{XY}(\tau)=Be^{jw_0\tau} $ for any $y(t)$
\end{block}


\end{frame}


% Blocks frame
\section{solution}
\begin{frame}{solution}
\begin{block}{}
If $R_x(\tau)=e^{jw_0\tau}$\\
then $S_x(\omega)=2\pi\delta(\omega-\omega_0)$
\end{block}
Hence the integral of $S_x(\omega)$ equals zero in any interval not including the point $\omega=\omega_0$\\
We know that the cross correlation $R_{XY}(\tau)$ satisfies the inequality
\begin{align}
R_{XY}^2(\tau)\le R_{XX}(0)R_{YY}(0)\nonumber
\end{align}
\end{frame}

\begin{frame}{}
Also the autocorrelation and autocovariance of $X[n]$ are given by
\begin{align}
R[n_1,n_2]=E\{x[n_1]x^*[n_2]\}\hspace{4mm} C[n_1,n_2]=R[n_1,n_2]-\eta[n_1]\eta^*[n_2]\nonumber
\end{align}
respectively where $\eta[n]=E\{x[n]\}$ that is mean of $x[n]$\\\vspace{5mm}
hence from the above statements, it follows that same is true for integral $S_{XY}(\omega)$.\\\vspace{5mm}
This shows that $S_{XY}(\omega)$ is a line at $\omega=\omega_0$ for any $y(t)$.

\end{frame}



\end{document}